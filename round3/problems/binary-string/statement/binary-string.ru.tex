\begin{problem}{Наука}{\textsl{standard input}}{\textsl{standard input}}{2 seconds}{256 mebibytes}{}

Маленький Артёмка решил заняться наукой. Он подался в химию и стал изучать кристаллические решётки. Артёмке интересны связи, которые образуются в них. 

В этой задаче можно представить себе кристаллическую решётку как строку длины $n$, состоящую из двух типов атомов. Артёмку интересуют только атомы первого типа. Будем называть \textit{$k$-устойчивой связью} подстроку, состоящую из $k$ последовательных атомов первого типа, ограниченных по бокам атомами второго типа.

Артёмка хочет посчитать ожидаемое число $k$-устойчивых связей для случайно сгенерированной строки длины $n$. Каждый атом случайно сгенерированной строки равновероятно и независимо от других может быть атомом первого или второго типа.

\InputFile
В первой строке записаны два целых числа $n$ и $k$ ($1 \le n \le 10^{18}$, $1 \le k \le 100$) --- длина строки и требуемая длина связей.

\OutputFile
Выведите одно вещественное число --- искомое математическое ожидание. Ответ будет засчитан, если абсолютная или относительная погрешность не превосходит $10^{-9}$.

\Examples

\begin{example}
\exmp{3 1
}{0.12500000000000000000
}%
\exmp{5 2
}{0.12500000000000000000
}%
\exmp{5 1
}{0.37500000000000000000
}%
\end{example}

\end{problem}
