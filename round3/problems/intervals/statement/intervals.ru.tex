\begin{problem}{Интервалы}{\textsl{standard input}}{\textsl{standard input}}{5 seconds}{256 mebibytes}{}

Маленький Артёмка хочет основать свой стартап. Он уверен в успехе! Сейчас он не может рассказать свою идею по понятным причинам, но ему нужна ваша помощь.

У Артёмки много планов и задач, а вот времени не хватает. Он просит вас решить одну из задач.

Вам дан массив чисел. Вам нужно найти количество таких различных пар $(l, r)$ ($l < r$), что среди чисел $a_l, a_{l+1}, \ldots, a_r$ найдутся два элемента на разных позициях, такие что разность между ними строго равна $d$. Более формально, должны существовать такие $i$ и $j$, что $l \le i, j \le r$, $i \neq j$ и $a_i - a_j = d$.


\InputFile
В первой строке вам даны два целых числа: $n$ и $d$ ($1 \le n \le 3 \cdot 10^5$, $-10^9 \le d \le 10^9$). В следующей строке записаны $n$ целых чисел $a_1, a_2, \ldots, a_n$, разделённых пробелами ($-10^9 \le a_i \le 10^9$).

\OutputFile
Выведите единственное целое число --- ответ на задачу.

\Examples

\begin{example}
\end{example}

\end{problem}
