\begin{problem}{Злые палиндромы}{\textsl{standard input}}{\textsl{standard input}}{6 seconds}{256 mebibytes}{}

Маленький Артёмка очень любит палиндромы и свою машину. Он любит свою машину так сильно, что даже купил дорогой цифровой замок для двери. Чтобы открыть замок, нужно ввести код --- целое число. Так как Артёмка любит палиндромы, он, разумеется, выбрал в качестве кода один из них. Одна беда --- он все время забывает правильное число.

Число называется палиндромом, если его десятичная запись без лишних ведущих нулей читается одинаково как слева направо, так и справа налево. Заметим, что отрицательные числа не являются палиндромами.

После очередного раза, когда Артёмка забыл код, и ему пришлось ехать на автобусе, он решил наконец-то сделать для себя подсказку. Но он не хочет, чтобы подсказка была очень лёгкой. Он решил написать на замке такое число $n$, что код к замку --- это $k$-й следующий для числа $n$ палиндром. Само число $k$ Артёмка решил запомнить.

Определим формально $k$-й следующий палиндром для числа $n$ при $k \ne 0$. Если $k > 0$, будем выписывать все числа-палиндромы, которые строго больше $n$, в порядке возрастания; ответом будет $k$-е выписанное число. Если же $k < 0$, будем выписывать все числа-палиндромы, которые строго меньше $n$, в порядке убывания; ответом будет $|k|$-е выписанное число.

Помогите Артёмке по известным $n$ и $k$ восстановить код.

\InputFile
В первой строке даны два целых числа $n$ и $k$ ($-10^{100\,000} < n < 10^{100\,000}$, $-10^9 \le k \le 10^9$, $k \ne 0$) --- подсказка, записанная Артёмкой, и запомненное им число. Заданные числа не содержат лишних ведущих нулей.

\OutputFile
Найдите $k$-й следующий палиндром для числа $n$. Если он не существует, выведите <<\texttt{-1}>>.

\Examples

\begin{example}
\exmp{15 1
}{22
}%
\exmp{0 1
}{1
}%
\exmp{-1 -1
}{-1
}%
\exmp{22 -1
}{11
}%
\end{example}

\end{problem}
