\begin{problem}{Тетраэдр}{\textsl{standard input}}{\textsl{standard input}}{1 second}{256 mebibytes}{}

Маленький Артёмка нашёл тетраэдр.
Артёмка поставил его на горизонтальный стол, поверхность которого совпадает с плоскостью $z = 0$, так, что координаты всех вершин тетраэдра оказались целыми.
После этого он отпустил тетраэдр, оставив его во власти силы тяжести, действующей вертикально вниз.

Артёмку интересует вопрос, будет ли тетраэдр в данном положении стоять устойчиво, стоять неустойчиво или упадёт набок.
Известно, что тетраэдр однородный, то есть масса любой его части пропорциональна объёму этой части.
Помогите ему ответить на этот вопрос.


\InputFile
Ввод состоит из четырёх строк; $i$-я из этих строк содержит три целых числа
$x_i$, $y_i$ и $z_i$ --- координаты $i$-й вершины тетраэдра.

При этом $-1000 \le x_i, y_i, z_i \le 1000$, $z_1 = z_2 = z_3 = 0$ и $z_4 > 0$. 

Гарантируется, что объём тетраэдра не равен нулю.


\OutputFile
Если тетраэдр будет стоять устойчиво, выведите ``\texttt{Standing}'',
если будет находиться в неустойчивом положении --- ``\texttt{Unstable}'',
а если упадёт набок --- ``\texttt{Falling}''.

\Examples

\begin{example}
\exmp{1 1 0
3 1 0
1 3 0
2 2 2
}{Standing
}%
\exmp{0 0 0
2 0 0
0 2 0
-2 -2 1
}{Unstable
}%
\exmp{1 1 0
3 1 0
1 3 0
10 2 2
}{Falling
}%
\end{example}

\end{problem}
