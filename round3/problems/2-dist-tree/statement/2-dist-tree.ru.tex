\begin{problem}{Расстояние}{\textsl{standard input}}{\textsl{standard input}}{2 seconds}{256 mebibytes}{}

Маленький Артёмка очень любит свою машину. Он любит кататься на ней каждый вечер по округе. Но машина Артёмки очень старая и не может проехать за раз больше двух миль.

Артёмка живёт в округе, где есть $n$ городов, соединённых $n - 1$ дорогой с двусторонним движением. Все города связаны между собой. Это значит, что существует путь между каждой парой городов, возможно, проходящий через несколько дорог. Длина каждой дороги равна одной миле.

Артёмка хочет узнать, сколько существует различных пар городов, таких что длина кратчайшего пути между ними строго равна двум милям. Две пары городов считаются различными, если существует город, содержащийся ровно в одной из этих пар.

\InputFile
В первой строке вам дано единственное целое число $n$ ($1 \le n \le 10^5$) --- число городов. Следующие $(n - 1)$ строк содержат описание дорог. В каждой строке записаны два числа $v_i$ и $u_i$ ($1 \le v_i, u_i \le n$), означающие, что между городами $v_i$ и $u_i$ есть дорога с двусторонним движением.

Гарантируется, что из любого города можно добраться по дорогам в любой другой.

\OutputFile
Выведите единственное число --- ответ на задачу.


\Examples

\begin{example}
\exmp{3
1 2
2 3
}{1
}%
\exmp{4
1 2
2 3
1 4
}{2
}%
\end{example}

\end{problem}
