\begin{problem}{Бургер-бар}{\textsl{standard input}}{\textsl{standard output}}{2 seconds}{256 mebibytes}{}

Заработав на разведении уток, Войтек решил открыть бургер-бар. Считается, что он является лучшим в городе, ну или одним из лучших...
Как минимум, это хорошее заведение, и уж во всяком случае не худшее. В бургер-баре много завсегдатаев... как минимум, Томек точно
является его завсегдатаем.

Томек рекомендовал бар Войтека некоторым из своих друзей; они спросили, сколько возможных дополнительных ингредиентов можно выбрать для
бургера. Точное число Томек не запомнил, однако у него в кармане нашлось несколько чеков. Каждый раз при покупке бургера Томек выбирал
некоторый набор дополнительных ингредиентов и оплачивал их суммарную стоимость (при этом оплачивается сам факт выбора дополнительного
ингредиента, а не его количество).

Зная сумму в злотых, которую Томек платил за дополнительные ингредиенты при каждом посещении, определите наименьшее возможное количество
различных дополнительных ингредиентов в баре Войтека.


\InputFile

Первая строка ввода содержит целое число $n$ ($1 \leq n \leq 20$) --- количество чеков в кармане у Войтека.
Вторая строка содержит $n$ целых чисел $a_1, \ldots, a_n$ ($1 \leq a_i \leq 50$) --- суммарную стоимость дополнительных ингредиентов на каждом из 
чеков.

\OutputFile

Выведите одно число --- наименьшее возможное количество различных дополнительных ингредиентов в ресторане Войтека. Обратите внимание, что 
цена каждого дополнительного ингредиента является неотрицательным целым числом.

\Examples
\begin{example}
\exmp{
5
42 1 3 3 6
}{
4
}%
\end{example}

\Note

В примере к условию возможен такой вариант: первый ингредиент стоит 1 злотый, второй и третий --- по 3 злотых, четвёртый --- 35 злотых.

\end{problem}
