\begin{problem}{Обработка данных}{\textsl{standard input}}{\textsl{standard output}}{2 seconds}{256 mebibytes}{}

Войтек работает программистом в компании Bytes, Inc. Он отвечает за сбор данных. В его распоряжении находится кластер из $n$
последовательно соединённых серверов. Сервер 1 соединён только с сервером 2, сервер 2 соединён с серверами 1 и 3 и так далее
до сервера $n$, который соединён только с сервером $n-1$.

Войтек узнал, что начальство собирается забрать для своих целей все сервера, на которых хранится мало данных.
Чтобы избежать этого, Войтек решил заполнить сервера данными. В процессе заполнения он может выбрать любой сервер и скопировать все
данные с него на один из серверов, с которым тот соединён. У Войтека есть время, чтобы провести $k$ операций подобного рода
до момента, когда начальство будет принимать решение о перераспределении.

Войтек хочет максимизировать объём данных, хранящийся на наименее заполненном сервере. Более формально,
пусть на $n$ серверах по завершении процесса хранится объём данных $s_1, \ldots, s_n$; требуется максимизировать $S=\min(s_1, \ldots, s_n)$.
Считается, что объём свободного места на каждом из серверов практически бесконечен.


\InputFile

Первая строка входа содержит целое число $k$ ($0 \leq k \leq 10^6)$ --- количество операций, которое может провести Войтек.

Вторая строка содержит целое число $n$ ($2 \leq n \leq 10\,000)$ --- количество сервреов в кластере.
Третья строка содержит $n$ целых чисел $a_1, \ldots, a_n$ ($1 \leq a_i \leq 10^8$) --- объём данных на соответствующих серверах перед
началом процесса копирования.

\OutputFile

Выведите одно число ---  максимальный объём данных, который будет храниться на наименее заполненном сервере после того, как Войтек проведёт
$k$ операций копирования. Поскольку ответ может быть очень большим, выведите его по модулю $10^9 + 7$.

\Examples
\begin{example}
\exmp{
5
5
42 1 3 3 6
}{
43
}%
\end{example}

\Note

    В примере из условия Войтек может сделать следующие операции копирования:
    \begin{itemize}
        \item с сервера $2$ на сервер $1$,
        \item с сервера $1$ на сервер $2$,
        \item с сервера $2$ на сервер $3$,
        \item с сервера $3$ на сервер $4$,
        \item с сервера $4$ на сервер $5$.
    \end{itemize}
    Распределение занятого места на серверах по окончании операции копирования: $43, 44, 47, 50$ и $56$, соответственно.


\end{problem}
