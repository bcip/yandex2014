\begin{problem}{Подстройка уток}{\textsl{standard input}}{\textsl{standard output}}{2 seconds}{256 mebibytes}{}

Войтек занимается разведением уток. Его сосед Томек пожаловался на то, что утки постоянно крякают... <<Ну и что?>> --- ответил Войтек,
после чего Томек достал рогатку и стал подбирать соответствующие камни. В результате соседи согласились на переговоры.

Как выяснилось, если кряканье каких-то двух и более уток идёт с одинаковой тональностью, то звук в данной тональности становится
монотонным и Томека не раздражает. То есть, если для каждой утки будет существовать как минимум ещё одна, крякающая в данной тональности,
то компромисс будет достигнут. Местный ветеринар-отоларинголог умеет изменять тональность, в которой крякают утки. За смену тональности
кряканья утки с $a$ на $b$ ветеринар берёт $|b-a|$ злотых.

В какую минимальную цену обойдётся Войтеку достижение компромисса?

\InputFile

Первая строка входя содержит целое число $n$ ($2 \leq n \leq 10^5$) --- количество уток у Войтека.
Вторая строка содержит $n$ целых чисел $a_1, \ldots, a_n$ ($1 \leq a_i \leq 10^8$) --- тональности, в которых крякают утки.

\OutputFile

Выведите наименьшую цену, которую должен заплатить Войтек для того, чтобы для каждой утки существовала как минимум одна, крякающая в той же
тональности.

\Examples
\begin{example}
\exmp{
5
42 1 3 3 6
}{
38
}%
\end{example}

\Note

Вариант решения --- сделать тональность кряканья второй утки равной 3 (за 2 злотых), а тональность кряканья последней утки равной 42 (за 36 злотых).

\end{problem}
