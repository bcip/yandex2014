\begin{problem}{Делёж денег}{\textsl{standard input}}{\textsl{standard output}}{2 seconds}{256 mebibytes}{}

Томек и Войтек решили выкопать колодец. В процессе рытья колодца они нашли коробку, заполненную дореформенными банкнотами.
Некоторые банкноты могли иметь одинаковые номиналы, все банкноты отличимы друг от друга по номерам. Учитывая, что обмен
дореформенных банкнот уже закончился, а ценности для коллекционеров такие банкноты пока что не представляют, соседи решили
поделить находку между собой следующим не вполне обычным образом. Томек берёт себе несколько банкнот (или не берёт ничего), после чего Войтек
берёт себе несколько банкнот (или не берёт ничего) таким образом, что $xor$ номиналов выбранных Войтеком банкнот совпадает с $xor$ номиналов
выбранных Томеком банкнот. Не выбранные Томеком или Войтеком банкноты остаются в коробке.

Сколько существует различных способов разделить находку указанным образом?

\InputFile

Первая строка входа содержит целое число $n$ ($1 \leq n \leq 10\,000$) --- количество банкнот. Вторая строка содержит
$n$ целых чисел $a_1, \ldots, a_n$ ($1 \leq a_i \leq 10\,000$) --- номиналы банкнот.

\OutputFile

Выведите одно число --- количетво способов выбрать по несколько (или ни одной) купюр так, чтобы $xor$ номиналов выбранных Томеком
купюр совпадал с $xor$ номиналов выбранных Войтеком купюр. Так как ответ может быть большим, выведите его по модулю
$10^9+7$.

\Examples
\begin{example}
\exmp{
5
42 1 3 3 6
}{
5
}%
\end{example}

\Note

В примере из условия возможны следующие варианты распределения:
\begin{itemize}
    \item $\{3\} : \{3\}$ (двумя способами)
    \item $\{3,3\} : \{\}$ 
    \item $\{\} : \{3,3\}$ 
    \item $\{\} : \{\}$ 
\end{itemize}

\end{problem}
