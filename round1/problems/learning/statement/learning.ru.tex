\begin{problem}{Обучение сложению}{\textsl{standard input}}{\textsl{standard output}}{2 seconds}{256 mebibytes}{}

Маленький Томек учится складывать. Учитель написал на доске последовательностть из $n$ целых чисел. Томек идёт слева направо и по 
очереди прибавляет числа к имеющейся у него сумме (которая изначально равна нулю).
Однако время от времени он путает порядок цифр. Фактически после каждого сложения он записывает сумму в случайном порядке,
опуская ведущие нули.

Например, если предыдущее значение суммы было равно $67$ и Томек прибавил $42$, в качестве новой суммы может оказаться записано
любое из чисел  $109$, $901$, $190$, $910$, $91$ или $19$.

Какую максимальную сумму Томек сможет получить в конце?

\InputFile

Первая строка входа содержит одно целое число $n$ ($1 \leq n \leq 5$) --- длину выписанной учителем последовательности.
Вторая строка содержит $n$ целых чисел $a_1, \ldots, a_n$ ($1 \leq a_i \leq 100$) --- элементы этой последовательности, перечисленные подряд,
начиная с самого первого.

\OutputFile

Выведите одно число --- максимальную сумму, которую сможет получить Томек.

\Examples
\begin{example}
\exmp{
5
42 1 3 3 6
}{
100
}%
\end{example}

\Note

В примере Томек может выполнить следующие действия:
    \begin{itemize}
        \item 0 + 42 = 42
        \item 42 + 1 = 43
        \item 43 + 3 = 46
        \item 46 + 3 = \textbf{94}
        \item 94 + 6 = 100
    \end{itemize}
Отметим, что в данной последовательности Томек переставил цифры суммы только один раз; в общем случае он может сделать любое количество перестановок.

\end{problem}
