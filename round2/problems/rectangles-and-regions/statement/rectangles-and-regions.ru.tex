\begin{problem}{Прямоугольники и области связности}{\textsl{standard input}}{\textsl{standard output}}{3 секунды}{256 мегабайт}

Маленький Петя очень любит прямоугольные таблицы. Недавно мама подарила ему таблицу, состоящую из $N$ строк и $M$ столбцов, каждая клетка которой покрашена либо в белый, либо в чёрный цвет. Больше чем таблицы Петя любит только играть с маленькой Машей. Маша предложила ему сыграть в игру с новой таблицей. Игра состоит из $Q$ раундов. В каждом раунде Маша выбирает в таблице прямоугольник, а Петя ей сообщает количество связных областей клеток одного цвета, которые находятся в этом прямоугольнике (см.~определение ниже). Поскольку дети пока умеют считать только до двух, \emph{все числа, которые больше двух, для них одинаковые} (смотрите примеры для дальнейшего разъяснения). Ваша задача --- помочь Пете ответить на все вопросы Маши.

Считается, что две клетки одного цвета находятся в одной связной области внутри заданного прямоугольника, если существует такой путь, начинающийся в первой клетке и заканчивающийся во второй, который удовлетворяет следующим условиям:
\begin{itemize}
\item Все клетки на пути окрашены в один цвет.
\item Любые две последовательные клетки этого пути имеют общую сторону.
\item Все клетки пути лежат в заданном прямоугольнике.
\end{itemize}

\InputFile
В первой строке записаны два целых числа $N$ и $M$ ($1 \le N \le 2000$, $1 \le M \le 2000$) --- количество строк и столбцов в таблице, соответственно. Следующие $N$ строк содержат по $M$ символов и задают Петину таблицу. Символ <<\texttt{1}>> соответствует чёрной клетке, а символ <<\texttt{0}>> --- белой.

Cледующая строка содержит целое число $Q$ ($1 \le Q \le 500\,000$) --- количество вопросов Маши. Каждая из следующих $Q$ строк содержит четыре целых числа $r_1$, $c_1$, $r_2$ и $c_2$ ($1 \le r_1 \le r_2 \le N$, $1 \le c_1 \le c2 \le M$) таких, что $(r_1, c_1)$ и $(r_2, c_2)$ --- координаты двух противоположных углов прямоугольника.

\OutputFile
Выведите $Q$ строк --- ответы на вопросы Маши в том же порядке, в котором они заданы на входе. \textbf{Если ответ на какой-либо вопрос больше или равен $3$, выведите вместо него число $0$.}

\Examples

\begin{example}
\exmp{4 5
01011
10101
01011
00001
5
1 1 2 3
3 1 4 3
3 3 4 5
1 5 4 5
1 4 4 5
}{0
2
2
1
0
}%
\exmp{5 5
11111
11111
11111
11101
11111
3
1 1 3 3
1 1 5 5
3 3 5 5
}{1
2
2
}%
\end{example}

\end{problem}
