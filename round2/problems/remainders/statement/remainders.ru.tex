\begin{problem}{Остатки}{\textsl{standard input}}{\textsl{standard output}}{2 секунды}{256 мегабайт}

Маленький Петя очень любит математику. Недавно на уроке математики он узнал про целочисленное деление с остатком. В качестве домашнего задания Петя получил такую задачу: даны числа $a_1$, $a_2$, $\dots$, $a_n$, требуется посчитать значение выражения 
$$(\dots((a_{1} \bmod a_{2}) \bmod a_{3}) \dots \bmod a_{n - 1}) \bmod a_{n}\text{.}$$

Здесь $a \bmod b$ означает взятие остатка от деления числа $a$ на число $b$. Петя записал сами числа $a_i$, но забыл, в каком именно порядке они были даны на доске. Поэтому он решил перебрать все $N!$ перестановок чисел $a_i$ и для каждой посчитать значение требуемого выражения, чтобы учительница сама выбрала нужную перестановку. Вскоре Петя понял, что перестановок может быть очень много, и он не успеет перебрать их все до следующего урока математики. К счастью, несмотря на большое количество перестановок, различных результатов выражения может быть не так и много. Поэтому Петя решил найти все числа, которые могут быть результатами вычисления выражения, записанного выше, для некоторой перестановки чисел $a_i$. Вам же предстоит всего лишь посчитать их количество.

\InputFile
Первая строка содержит целое число $N$ ($2 \le N \le 100$) --- количество чисел, которые дала учительница. Следующая строка содержит $N$ целых положительных чисел, разделённых пробелами. Все эти числа не превосходят $3 \cdot 10^5$.

\OutputFile
Выведите количество различных значений выражения из задачи Пети.

\Examples

\begin{example}
\exmp{4
5 6 7 8
}{4
}%
\exmp{3
10 7 10
}{3
}%
\exmp{5
34 20 199 22 135
}{16
}%
\end{example}

\Note
В первом примере возможными ответами являются числа $1$, $2$, $3$ и $5$.

Во втором примере возможны всего три перестановки чисел, и все они дают разные ответы:
$$(7 \bmod 10) \bmod 10 = 7\text{,}$$
$$(10 \bmod 7) \bmod 10 = 3\text{,}$$
$$(10 \bmod 10) \bmod 7 = 0\text{.}$$

\end{problem}
