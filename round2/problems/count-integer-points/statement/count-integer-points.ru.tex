\begin{problem}{Точки}{\textsl{standard input}}{\textsl{standard output}}{2 секунды}{256 мегабайт}

Даны три попарно неколлинеарных вектора $\vec{a} = (a_x, a_y)$, $\vec{b} = (b_x, b_y)$, $\vec{c} = (c_x, c_y)$ и натуральное число $N$. Посчитайте количество различных точек $p$ таких, что $p = u_a \cdot \vec{a} + u_b \cdot \vec{b} + u_c \cdot \vec{c}$, где числа $u_a$, $u_b$ и $u_c$ целые и $0 \le u_a, u_b, u_c \le N - 1$. Две точки $p = (p_x, p_y)$ и $q = (q_x, q_y)$ считаются различными, если $p_x \ne q_x$ или $p_y \ne q_y$.

\InputFile
В первой строке задано целое число $N$ ($1 \le N \le 2000$). Во второй строке перечислены через пробел шесть целых чисел: $a_x$, $a_y$, $b_x$, $b_y$, $c_x$, $c_y$ ($1 \le a_x, a_y, b_x, b_y, c_x, c_y \le 1000$).

\OutputFile
В единственной строке выведите искомое количество точек.

\Examples

\begin{example}
\exmp{5
2 1 1 2 3 3
}{61
}%
\exmp{5
10 1 5 5 1 10
}{125
}%
\exmp{2
2 3 1 2 1 1
}{7
}%
\end{example}

\end{problem}
