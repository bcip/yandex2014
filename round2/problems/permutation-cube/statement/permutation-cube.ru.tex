\begin{problem}{Permutation Cube}{\textsl{standard input}}{\textsl{standard output}}{3 секунды}{256 мегабайт}

Девочка Ксюша недавно установила себе на компьютер игру <<Permutation Cube>>. Правила игры таковы. В начале раунда генерируются три перестановки чисел $1$, $2$, $\dots$, $N$: $X$, $Y$ и $Z$. Во время каждого хода игрок может сделать одно из двух действий:
\begin{itemize}
\item Выставить курсор в произвольную точку $(u, v, t)$, где $1 \le u, v, t \le N$, заплатив штраф в один тугрик.
\item Переставить курсор бесплатно в точку $(X_u, Y_v, Z_t)$, где $(u, v, t)$ --- его текущее положение.
\end{itemize}

Целью игры является посетить курсором все $N^3$ точек (точки $(u, v, t)$ такие, что $1 \le u, v, t \le N$). Считается, что изначально курсор не выставлен ни в одну точку поля. Найдите минимальное число тугриков, которое придется заплатить Ксюше, чтобы пройти игру.

Напомним, что \emph{перестановка} --- это упорядоченный набор чисел $1$, $2$, $\dots$, $N$, в котором каждое из них встречается ровно один раз.

\InputFile
В первой строке входных данных содержится целое число $N$ --- размер игрового поля ($1 \le N \le 30\,000$). В следующих трёх строках, по одной в строке, заданы перестановки $X_i$, $Y_i$, $Z_i$. Гарантируется, что каждая из этих строк корректно задаёт перестановку чисел $1$, $2$, $\dots$, $N$. Элементы каждой перестановки разделены пробелами.

\OutputFile
Необходимо вывести единственное число --- минимальное количество тугриков, которое необходимо заплатить для прохождения игры <<Permutation Cube>>.

\Examples

\begin{example}
\exmp{3
1 2 3
3 1 2
2 1 3
}{6
}%
\exmp{3
1 3 2
2 3 1
3 1 2
}{6
}%
\end{example}

\end{problem}
