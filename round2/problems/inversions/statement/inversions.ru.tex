\begin{problem}{Инверсии}{\textsl{standard input}}{\textsl{standard output}}{2 секунды}{256 мегабайт}

Маленький Петя очень любит перестановки. Больше чем перестановки он любит только играть с маленькой Машей. Скоро у неё день рождения, на который Петя решил ей подарить перестановку. Он знает, что любимое число Маши --- это $K$. Поэтому перестановка, которую Петя будет дарить, должна иметь ровно $K$ инверсий (см.~определение ниже). Среди всех перестановок с $K$ инверсиями Петя хочет выбрать такую, которая состоит из наименьшего количества элементов, а среди них --- лексикографически минимальную. Помогите ему найти требуемую перестановку.

\emph{Перестановка} --- это упорядоченный набор чисел $1$, $2$, $\dots$, $N$, в котором каждое из них встречается ровно один раз. Число, стоящее на позиции $i$ в перестановке $\pi$, будем обозначать как $\pi(i)$.

\emph{Инверсией} в перестановке $\pi$ чисел $1$, $2$, $\dots$, $N$ называется всякая пара индексов $(i, j)$ такая, что $1 \le i < j \le N$ и $\pi(i) > \pi(j)$.

Считается, что перестановка $\pi$ лексикографически меньше перестановки $\sigma$, если для некоторого $j$ от 1 до $N$ выполняются следующие два свойства:
\begin{itemize}
\item $\pi(j) < \sigma(j)$;
\item $\pi(i) = \sigma(i)$ для всех $i$ от $1$ до $j - 1$, включительно.
\end{itemize}

\InputFile
На входе записано целое число $K$ ($0 \le K \le 1\,000\,000\,000$) --- требуемое количество инверсий в перестановке.

\OutputFile
Первая строка должна содержать целое положительное число $N$ --- количество элементов в искомой перестановке. Следующая строка должна содержать $N$ чисел, разделённых пробелами --- искомую перестановку.

\Examples

\begin{example}
\exmp{0
}{1
1
}%
\exmp{1
}{2
2 1
}%
\exmp{2
}{3
2 3 1
}%
\exmp{3
}{3
3 2 1
}%
\end{example}

\end{problem}
