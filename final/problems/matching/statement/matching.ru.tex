\begin{problem}{Недополняемое паросочетание}{stdin}{stdout}{1 секунда}{512 мегабайт}

Артёмка очень любит паросочетания.

\textit{Паросочетание} в неориентированном графе --- это набор попарно несмежных рёбер.

Паросочетание называется \textit{недополняемым}, если невозможно добавить в паросочетание ещё одно ребро, не удаляя при этом ни одно из уже включённых рёбер.

Неориентированный граф называется \textit{двудольным}, если его вершины можно разбить на два множества так, чтобы любое ребро графа соединяло вершины из разных множеств.

Артёмке дан двудольный граф. Его задача --- найти количество недополняемых паросочетаний в нем по модулю $1\,000\,000\,007$ ($10^9 + 7$). Ваша задача --- помочь ему.

\InputFile
В первой строке записано три целых числа $n_1$, $n_2$ и $m$ ($1 \le n_1 \le 10$, $1 \le n_2 \le 100$, $0 \le m \le n_1 \cdot n_2$) --- количества вершин в двух долях графа и количество рёбер, соответственно. В следующих $m$ строках содержится описание рёбер графа. В каждой строке записано по два целых числа $v_1$ и $v_2$ ($1 \le v_1 \le n_1$, $1 \le v_2 \le n_2$), означающих, что очередное ребро соединяет вершину $v_1$ первой доли и вершину $v_2$ второй доли. Гарантируется, что граф не содержит кратных рёбер.

\OutputFile
Выведите одно число --- количество недополняемых паросочетаний по модулю $1\,000\,000\,007$ ($10^9 + 7$).

\Examples

\begin{example}
\exmp{2 3 4
1 1
1 2
1 3
2 2
}{3
}%
\exmp{2 2 4
1 1
1 2
2 1
2 2
}{2
}%
\end{example}

\end{problem}
