\begin{problem}{Строчная задача}{stdin}{stdout}{4 секунды}{512 мегабайт}

Поисковые запросы --- что может быть лучше? Поле для ввода вопроса, кнопка <<найти>> --- и весь интернет выстраивается в упорядоченный список ответов. Миллионы поисковых запросов сохраняются каждый день, чтобы в будущем можно было отвечать на них быстрее и лучше, но этого мало! Важно помочь пользователю быстрее вводить сам поисковый запрос. Например, можно подсказывать популярные или интересные пользователю варианты запросов, упорядочивая их в соответствии с предпочтениями. Но что делать, если мы ничего не знаем о том, кто задаёт вопрос? 

Вы работаете над алгоритмом, который будет предлагать две подсказки в лексикографическом порядке. Ну, почти в лексикографическом.

Известно, что пользователи могут допустить опечатки при вводе запроса. Скажем, что строка $A$ \emph{почти меньше} строки $B$, если строка $A$ лексикографически меньше строки $B$, или же если в строке $A$ можно заменить одну букву на какую-то другую маленькую букву английского алфавита так, чтобы полученная строка оказалась лексикографически меньше строки $B$. 

Вам даны $N$ слов-подсказок. Посчитайте количество пар слов, первое из которых почти меньше второго. Количество пар нужно вычислять с учётом порядка: если два слова являются взаимно \emph{почти меньшими}, то учитывайте обе пары.


\InputFile
В первой строке содержится целое число $N$ ($1 \le N \le 10^6$) --- количество слов-подсказок. 
Следующие $N$ строк содержат слова, состоящие только из маленьких букв английского алфавита. Суммарная длина всех слов не превосходит $10^6$.

\OutputFile
Выведите искомое количество пар.

\Examples

\begin{example}
\exmp{3
a
b
c
}{4
}%
\exmp{2
aab
aba
}{2
}%
\exmp{2
aa
aaa
}{1
}%
\end{example}

\end{problem}
