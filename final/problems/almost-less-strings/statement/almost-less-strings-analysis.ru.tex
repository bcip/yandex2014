\begin{problem}{Строчная задача}
{\textsl{standard input}}{\textsl{standard output}}
{4 seconds}{512 mebibytes}{}

Рассмотрим пару строк $s$ и $w$. Строка $s$ почти меньше строки $w$ тогда и только тогда, когда минимальная лексикографически строка $s^{min}$, которую можно получить из $s$ заменой одной буквы, строго меньше $w$. 

Однако, как легко видеть, $s^{min}$ получается из строки $s$ заменой первого символа не равного `\texttt{a}' на символ `\texttt{a}'. Если строка $s$ состоит только из символов `\texttt{a}', то $s = s^{min}$.

Рассмотрим строку $s_i$, и найдем $C_i$~--- количество строк $s_j$ (возможно $i=j$) таких, что $s_i^{min} < s_j$. Тогда к ответу нужно прибавить $C_i - 1$, если $s_i^{min} \ne s_i$, и $C_i$ в противном случае. 

Для нахождения величин $C_i$ можно воспользоваться двумя способами:
\begin{itemize}
\item отсортировать строки $s_i$ и находить $C_i$ двоичным поиском;
\item добавить все строки в \texttt{trie}, и находить $C_i$ спуском по нему.
\end{itemize}

\end{problem}
