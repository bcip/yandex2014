\begin{problem}{1024 Stack Edition}
{\textsl{standard input}}{\textsl{standard output}}
{1 second}{512 mebibytes}{}

Для начала рассмотрим случай, когда $N = 0$. Пусть $f(i)$ --- это ожидаемое наименьшее количество монет, которое потребуется потратить, чтобы собрать число $2^i$. Приведем формулы для подсчета $f(i)$ в случае, если $p < 100$. Если же $p = 100$, то эти же формулы адаптируются тривиальным образом.

\begin{enumerate}

\item $f(0) = 1.0 / p - 1$;

\item $f(1) = min (f(0) \cdot 2,~1 / (1 - p) - 1,~1 - p)$;

\item $f(i) = 2 \cdot f(i - 1),~i > 1$.

\end{enumerate}

В случае с $N = 0$ ответом на задачу будет $f(10)$.

Теперь предположим, что $N > 0$. Обозначим через $g(i, j)$ ожидаемое наименьшее количество монет, которые нам нужно будет потратить, чтобы из последних (верхних) $i$ элементов стека получить только одно число $2^j$. Подсчет $g(i, *)$ для фиксированного i будет происходить в 2 этапа (через $S(i)$ обозначим двоичный логарифм $i$-го сверху элемента стека):

\begin{enumerate}

\item Посчитаем $g(i, S(i)) = 1 + min( g(i - 1, *) ),~i > 1$. Этот случай соответствует тому, что из всего, что выше элемента $i$, мы получаем некоторое число, после чего удаляем его.

Также посчитаем $g(i, S(i) + 1) = g(i - 1, S(i))$. Этот случай соответствует тому, что мы из всех элементов, расположенных выше $i$, собираем число $2^{S(i)}$, после чего объединяем его с текущим элементом, который также равен $2^{S(i)}$. Для всех $j$, отличных от $S(i)$ и $S(i) + 1$, положим $g(i, j) = +\inf$.

\item На первом этапе мы ``разобрались'' с числом $S(i)$. Прежде чем переходить к $g(i + 1, *)$ и числу $S(i + 1)$, мы можем произвести некоторые действия с $i$-м элементом стека, превратив его в любой другой элемент. Чтобы это учесть в динамике, пересчитаем $g(i, *)$ следующим образом: 

\begin{itemize}

\item $g(i, 0) = min( g(i, 0), min ( g(i, *) ) + 1.0 / p)$ --- мы либо используем уже найденное значение $g(i, 0)$, либо берем любое число, удаляем его и ставим туда $1 = 2^0$.
    
\item $g(i, j) = min( g(i, j), min ( g(i, *) ) + 1 + f(j), g(i, j - 1) + f(j - 1) ),~j > 0$. Здесь мы либо удаляем стоявшее на месте $i$ число и затем за $f(j)$ монет ставим туда число $2^j$, либо ставим за $g(i, j - 1)$ шагов число $2^{j-1}$, собираем за $f(j-1)$ шагов еще $2^{j-1}$ и объединяем их.

\end{itemize}
\end{enumerate}

Ответом на задачу будет $g(N, 10)$. Следует обратить внимание, что в промежуточных вычислениях следует также считать $g(*, 11)$. Сложность решения $O(NR)$, где $R$ --- максимальная степень двойки, которая присутствует во входных данные (в данном случае $R \le 10$).

\end{problem}
