\begin{problem}{1024 Stack Edition}{stdin}{stdout}{1 секунда}{512 мегабайт}

Недавно девочка Ксюша купила в онлайн-магазине приложение <<1024 Stack Edition>> --- пошаговую аркаду. Как можно догадаться из названия, игра разворачивается вокруг некоторого стека, наполненного степенями двойки. По правилам игры на каждом ходу игрок может выполнить одно из трёх действий над стеком:
\begin{itemize}
\item Добавить случайное число на вершину стека.
\item Убрать число с вершины стека.
\item Если два числа на вершине стека равны, то заменить их на их сумму.
\end{itemize}

Если игрок добавляет случайное число на вершину стека, то с вероятностью $p$ это число будет единицей, а с оставшейся вероятностью $(1 - p)$ --- двойкой, независимо от других сгенерированных случайных чисел. Если игрок выбирает убрать число с вершины стека, он платит штраф в одну монету. Обратите внимание, что все остальные операции, кроме удаления числа с вершины стека, не приводят к уплате штрафа. Для того чтобы выиграть, игроку необходимо привести стек в такое состояние, что в нём находится лишь один элемент --- число $1024$. При этом чем меньше штрафа он заплатит --- тем лучше.

Как-то раз Ксюша оставила свой телефон без присмотра, а по возвращении обнаружила, что кто-то играл в <<1024 Stack Edition>>.  Ксюше не интересно, кто это был и каковы были его мотивы. Ей интересно узнать, каково математическое ожидание штрафа, который лично ей придётся заплатить при оптимальной игре, если она доиграет партию. Ваша задача, как вы могли уже догадаться, посчитать это число.


\InputFile
В первой строке ввода содержатся два целых числа --- исходный размер стека $N$ ($0 \le N \le 100\,000$) и вероятность p ($0 < p \le 100$), указанная \textbf{в процентах}. Во второй строке через пробел указаны $N$ целых чисел $s_i$ ($s_i = 2^h$, где $h$ целое и $0 \le h \le 10$), которые находятся в стеке, начиная с самых <<глубоких>> элементов (вершина стека --- это последнее число строки). Обратите внимание, что стек может быть пустым, тогда во второй строке не содержится ни одного числа. Также заметьте, что размер стека во время игры ничем не ограничен, задан лишь начальный его размер.

\OutputFile
В единственной строке следует вывести искомое число --- математическое ожидание количества монет, которое придётся заплатить Ксюше, доигрывая заданную во вводе партию. Абсолютная или относительная погрешность ответа не должна превосходить $10^{-6}$.

\Examples

\begin{example}
\exmp{10 50
512 256 128 64 32 16 8 4 2 1
}{1.000000000
}%
\exmp{9 50
512 256 128 64 32 16 8 4 2
}{0.500000000
}%
\end{example}

\end{problem}
