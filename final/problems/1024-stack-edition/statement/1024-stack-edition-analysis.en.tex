\begin{problem}{1024 Stack Edition}
{\textsl{standard input}}{\textsl{standard output}}
{1 second}{512 mebibytes}{}

First consider the case when $N=0$. Let $f(i)$ be the expected minimal number
of coins needed to construct $2^i$. Then in case when $p<100$ we can 
use the following formulae to calculate $f(i)$:
\begin{enumerate}

\item $f(0) = 1 / p - 1$;

\item $f(1) = min (f(0) \cdot 2,~1 / (1 - p) - 1,~1 - p)$;

\item $f(i) = 2 \cdot f(i - 1),~i > 1$;

\end{enumerate}

These formulae can be easily tuned for the case when $p=100$. Thus, if $N = 0$ the answer will be $f(10)$.

Now consider $N > 0$. 

Let $g(i, j)$ be the expected minimal number of coins needed to
construct only one number $2^j$ from the top $i$ numbers of the stack. 
To count $g(i, *)$ for fixed $i$, we will use the following 2-step procedure;
here $S(i)$ stands for the binary logarithm of the $i$-th number from the top of stack.

\begin{enumerate}

\item Calculate $g(i, S(i)) = 1 + \min( g(i - 1, *) )$ for $i > 1$.
This represents a situation when we construct some number from
the topmost $i-1$ one and then delete it.
Also calculate $g(i, S(i) + 1) = g(i - 1, S(i))$.
This formula represents a situation when we construct a number $2^{S(i)}$
from the topmost $i-1$ one and then merge it
with the current number, which is also equal to $2^{S(i)}$.
For all $j$ other than $S(i)$ and $S(i) + 1$, let $g(i, j) = +\inf$.

\item On the previous stage we've processed the number $S(i)$.
Before switching
to $g(i + 1, *)$ and the number $S(i + 1)$ we can do something
with the $i$'th number on the stack, constructing some other number from it.
So we should recalculate $g(i, *)$ in the following way: 

\begin{itemize}

\item $g(i, 0) = \min( g(i, 0), \min (g(i, *)) + 1 / p)$,
i.e. we either use $g(i, 0)$ or delete the current number
and put $1 = 2^0$ instead.
    
\item $g(i, j) = \min( g(i, j), \min (g(i, *)) + 1 + f(j)$,
$g(i, j - 1) + f(j - 1) )$ for $j > 0$.
Here we can either delete the $i$'th number and replace it with 
$2^j$ using $f(j)$ coins or use $g(i, j - 1)$ steps to
create one $2^{j-1}$, then $f(j-1)$ steps to create
another $2^{j-1}$ and then merge them.

\end{itemize}
\end{enumerate}

The answer to our problem will be equal to $g(N, 10)$.
Note that during the process of calculation of $g(*, *)$,
we need to calculate $g(*, 11)$ as well.
The complexity of our solution is $O(NR)$ where $R$ is the highest
power of two present in the input data
(due to statement limitations $R \le 10$).

\end{problem}
