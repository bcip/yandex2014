\begin{problem}{Музыкальный мир}{stdin}{stdout}{1 секунда}{512 мегабайт}

В обновлённом сервисе Яндекс.Музыки музыкальные рекомендации стали ещё лучше, чем раньше! Они учитывают каждый прослушанный пользователем трек с самого начала использования сервиса, и в зависимости от уже прослушанных --- увеличивают пул рекомендаций. Плеер включает пользователю случайный из рекомендованных треков, и так целый день.

Но сервера Яндекса такие быстрые, что могут строить рекомендации не только для того трека, которую сейчас слушает пользователь, но и для всех вариантов случайно выбранных следующих треков.

Пользователь выбирает первый трек, а дальше алгоритм автоматически строит рекомендации на $N$ песен вперёд. Из каждой песни $i$-го поколения может получиться не более $K_i$ ($1 \le K_i \le 5$) новых песен в $(i + 1)$-м поколении, причём известна вероятность $p_{i, j}$ того, что на основании песни $i$-го поколения будет предсказано $j$ новых песен для каждого $j$ от $0$ до $K_i$. Предсказания на основании каждой отдельной песни строятся независимо от других песен того же поколения.

Так как экспериментальное исследование производилось на жанре инди, все рекомендованные треки различны. Для того, чтобы измерить разнообразие сбора статистики о похожести песен, похожесть измеряется для каждой пары треков. Чтобы помочь разработчикам Яндекса оценить затраты по памяти для этого измерения, вычислите математическое ожидание количества пар треков в пуле рекомендаций в $(N + 1)$-м поколении.

\InputFile
В первой строке задано число поколений $N$ ($1 \le N \le 10^5$), на которые производится предподсчёт.

В следующих $N$ строках задано распределение вероятностей числа предсказанных дорожек для каждого поколения.
В $i$-й из этих строк задано число $K_i$ ($1 \le K_i \le 5$) --- максимальное число возможных предсказаний для $i$-го поколения.
После него дано $(K_i + 1)$ целое число $a_{i, j}$ ($1 \le a_{i, j} \le 1000$) --- ненормированная вероятность того, что у песни $i$-го поколения будет $j$ предсказаний для $(i + 1)$-го поколения. По ним вероятности $p_{i,j}$ вычисляются как $\frac{a_{i,j}}{\sum\limits_{t=0}^{k}{a_{i,t}}}$.

Первым поколением является первая песня, вторым --- песни, рекомендованные из-за первой песни, третьим --- песни, рекомендованные из-за каждой из песен второго поколения, и так далее.

\OutputFile
Так как ответ на задачу может быть достаточно большим, посчитайте его в виде несократимой рациональной дроби $\frac{A}{B}$ и выведите значение $(A \cdot B^{-1}) \bmod (10^{9} + 7)$. Здесь $B^{-1}$ --- обратное к числу $B$ по модулю $10^{9} + 7$. Ограничения на входные данные гарантируют, что знаменатель дроби $B$ не будет делиться на $10^{9} + 7$, поэтому это выражение корректно определено.


\Examples

\begin{example}
\exmp{2
2 1 1 1
2 1 1 1
}{666666672
}%
\exmp{2
2 1 1 1
2 1 1 2
}{520833338
}%
\exmp{2
2 1 1 2
2 1 1 1
}{916666674
}%
\end{example}

\Note
В первом тесте из примера оба поколения порождают от $0$ до $2$ песен с вероятностями $\frac{1}{3}$.
В третьем поколении будет:
\begin{itemize}
\item $0$ песен с вероятностью $\frac{13}{27}$,
\item $1$ песня с вероятностью $\frac{5}{27}$,
\item $2$ песни с вероятностью $\frac{2}{9}$,
\item $3$ песни с вероятностью $\frac{2}{27}$,
\item $4$ песни с вероятностью $\frac{1}{27}$.
\end{itemize}

В таком случае искомое математическое ожидание равно 
$\frac{2}{9}\binom{2}{2} + \frac{2}{27}\binom{3}{2} + \frac{1}{27}\binom{4}{2} = \frac{2}{3}$.

На второй тест ответ $\frac{49}{48}$, на третий --- $\frac{11}{12}$.

\end{problem}
