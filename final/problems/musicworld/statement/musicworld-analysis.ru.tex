\begin{problem}{Музыкальный мир}
{\textsl{standard input}}{\textsl{standard output}}
{1 second}{512 mebibytes}{}

Обозначим как $Y_i$ случайную величину,
равную количеству песен в $i$-м поколении.

Предлагаемое решение задачи основано на понятии производящей функции.
Если вы не знакомы с ним, ознакомиться можно, к примеру, в статье Википедии

\verb+http://ru.wikipedia.org/wiki/Производящая_функция_последовательности+

Рассмотрим производящие функции заданных во входе случайных величин
$\psi_i(x) = \sum\limits_{j=0}^{K_i}{p_{ij}x^j}$,
а также производящие функции случайных величин $Y_i$
$\varphi_i(x) = \sum\limits_{j=0}^{\infty}{\mathbb{P}\{Y_i=j\}x^j}$.

У нас есть явный вид всех $\psi_i$, а также явный вид $\varphi_1(x) = x$.
Научимся выражать $\varphi_{i+1}$ через $\psi_{i}$ и $\varphi_{i}$.

В случае, если бы мы точно знали, что $Y_i = k$,
то можно было бы утверждать, что $\varphi_{i+1}(x) = \psi_i(x)^k$.
Но, так как $Y_i$ случайная величина, то искомая производящая функция
есть сумма производящих функций, соответствующих разным значениям,
с весами, равными вероятностям этих значений, то есть
$$\sum\limits_{j=0}^{\infty}{\mathbb{P}\{Y_i=j\}\psi_i(x)^j} =
\varphi_i(\psi_i(x))\text{.}$$

Так как максимальное значение искомой величины очень большое,
вычислить явный вид производящей функции для $Y_{n+1}$
не представляется возможным. 
Однако он нам и не нужен.
Несложно заметить, что величина, которую необходимо посчитать
$$\mathbb{E}\frac{Y_i(Y_i-1)}{2} =
\sum\limits_{j=0}^{\infty}{\mathbb{P}\{Y_i=j\}\frac{j(j-1)}{2}} =
\frac{1}{2}\varphi_i''(1)\text{.}$$

Научимся пересчитывать значение, первую и вторую производную функции
$\varphi_{i+1}(x)$ в (1) через предыдущие функции.
Напомню, что $\varphi_i(1) = \psi_i(1) = 1$.
Как мы знаем, $\varphi_{i+1}(x) = \varphi_i(\psi_i(x))$.
Тогда $$\varphi'_{i+1}(x) = \varphi'_{i}(\psi_i(x))\psi_{i}'(x) \text{ и}$$
$$\varphi''_{i+1}(x) = \varphi''_{i}(\psi_i(x))\psi_{i}'(x)^2 +
\varphi'_{i}(\psi_i(x))\psi_{i}''(x)\text{.}$$

При подстановке $x = 1$ эти формулы упрощаются и принимают вид
$$\varphi'_{i+1}(1) = \varphi'_{i}(1)\psi_{i}'(1) \text{и}$$
$$\varphi''_{i+1}(1) = \varphi''_{i}(1)\psi_{i}'(1)^2 +
\varphi'_{i}(1)\psi_{i}''(1)\text{.}$$

Также несложно понять, что вычисления можно проводить сразу по модулю $10^9+7$,
заменив исходные вероятности на соответствующие значения по модулю.

\end{problem}
