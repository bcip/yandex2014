\begin{problem}{Musical World}
{\textsl{standard input}}{\textsl{standard output}}
{1 second}{512 mebibytes}{}

Author's solution is based on idea of the \emph{generating function}.
If you are not familiar with this idea, you can read, for example,

\verb+http://en.wikipedia.org/wiki/Generating_function+

Let $Y_i$ be the random variable equal to number of tracks in $i$-th
generation.

Consider generating functions of random variables  
$\psi_i(x) = \sum\limits_{j=0}^{K_i}{p_{ij}x^j}$ given in the input
and generating functions of random variables $Y_i$ 
$\varphi_i(x) = \sum\limits_{j=0}^{\infty}{\mathbb{P}\{Y_i=j\}x^j}$.

We have $\psi_i$ in explicit form, and we know that $\varphi_1(x) = x$. 
Let us try to express $\varphi_{i+1}$ using $\psi_{i}$ and $\varphi_{i}$.

In case when $Y_i = k$, $\varphi_{i+1}(x) = \psi_i(x)^k$.
But, because variable $Y_i$ is random, the required generating function
is weigted the sum of generating functions for differrent values
with weights equal to probabilities of thise values, i.e. 
$$\sum\limits_{j=0}^{\infty}{\mathbb{P}\{Y_i=j\}\psi_i(x)^j} =
\varphi_i(\psi_i(x))\text{.}$$

Because the required value can be very big, it is impossible to find
the generating function for $Y_(n+1)$ explicitly.
But let us take into account that

$$\mathbb{E}\frac{Y_i(Y_i-1)}{2} =
\sum\limits_{j=0}^{\infty}{\mathbb{P}\{Y_i=j\}\frac{j(j-1)}{2}} =
\frac{1}{2}\varphi_i''(1)\text{.}$$

Let us express the value and also first and second derivatives
of $\varphi_{i+1}(x)$ from (1) using previous functions.
We know that $\varphi_i(1) = \psi_i(1) = 1$ and
$\varphi_{i+1}(x) = \varphi_i(\psi_i(x))$.
Then $$\varphi'_{i+1}(x) = \varphi'_{i}(\psi_i(x))\psi_{i}'(x)\text{ and}$$
$$\varphi''_{i+1}(x) = \varphi''_{i}(\psi_i(x))\psi_{i}'(x)^2 +
\varphi'_{i}(\psi_i(x))\psi_{i}''(x)\text{.}$$

After substitution of $x = 1$, those equations can be simplified
to the following form:
$$\varphi'_{i+1}(1) = \varphi'_{i}(1)\psi_{i}'(1)\text{ and}$$
$$\varphi''_{i+1}(1) = \varphi''_{i}(1)\psi_{i}'(1)^2 +
\varphi'_{i}(1)\psi_{i}''(1)\text{.}$$

Also is easy to see that all calculations can be done using modulo $10^9+7$,
replacing the given probabilities with appropriate values using this modulo.

\end{problem}
