\begin{problem}{Кубик}{stdin}{stdout}{1 секунда}{512 мегабайт}

Дан куб, граням которого поставлены в соответствие различные целые числа от $1$ до $6$.

Также дано гексамино --- фигура из шести клеток на клетчатой плоскости, связная по стороне. Клеткам гексамино также поставлены в соответствие различные целые числа от $1$ до $6$.

Определите, можно ли из заданного гексамино сделать куб, который можно будет повернуть так, чтобы нумерация его граней совпала с нумерацией граней заданного куба.
Разрешается сгибать гексамино по границам клеток, но не разрезать его.

\InputFile
В первой строке ввода заданы шесть попарно различных целых чисел от $1$ до $6$ --- нумерация граней куба, перечисленных в следующем порядке: передняя, задняя, левая, правая, верхняя, нижняя.

В последующих строках задаётся гексамино в виде массива из $K \times N$ клеток: каждая из последующих $K$ строк содержит ровно $N$ цифр от $0$ до $6$ без пробелов. Нулям соответствуют пустые клетки, ненулевым цифрам --- соответствующие им клетки гексамино. 

Гарантируется, что образованная ненулевыми цифрами фигура связна по стороне, что
каждая из ненулевых цифр от $1$ до $6$ встречается ровно один раз, и что любая из $K$ строк и любой из $N$ столбцов массива содержит хотя бы одну ненулевую цифру.


\OutputFile
Выведите <<\texttt{Yes}>>, если сложить куб, нумерация граней которого совпадает с заданной, возможно, и <<\texttt{No}>> в противном случае.


\Examples

\begin{example}
\exmp{1 2 3 4 5 6
0100
3546
0200
}{Yes
}%
\exmp{1 6 2 5 3 4
1020
3546
}{No
}%
\end{example}

\end{problem}
