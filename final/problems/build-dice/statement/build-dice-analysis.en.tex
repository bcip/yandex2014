\begin{problem}{Dice}
{\textsl{standard input}}{\textsl{standard output}}
{1 second}{512 mebibytes}{}

The key point in any solution is the following:
hexomino can be folded in two ways (``inside'' and ``outside''),
and it will generate two dice which can be converted one into another
by swapping two numbers fitting to any pair of opposite faces.
Taking into account that rotation of the cube by 180 degrees swaps
numbers on two pairs of opposite faces, we can say that we do not need
to consider more than 2 orientations. 

After making the above observation, the problem may be solved in several ways.

One of them is the following. 

\begin{enumerate}
\item Put the cube on the hexomino such as the bottom face of the cube
will stay at the cell labeled by the same number as this face. 

\item For any of four possible rotations of the cube, try to roll it to
all other elements of hexomino; the cube can be rolled on
some cell if and only if this cell is labeled by the same number
as the bottom face of the cube after the roll.

\item If at least for one rotation all 6 cells of hexomino are visited,
the answer is ``\texttt{Yes}''. 

\item Otherwise, we will swap numbers between left and right faces and then
repeat from step 2. If the answer ``\texttt{Yes}'' is not reached,
then the answer is ``\texttt{No}''.
\end{enumerate}

Another way is to precalculate all 11 unfoldings of the cube and then to
rotate the given hexomino; if for some rotation or mirroring, the shapes
coincide, check if pairs of numbers on opposite sides fit into
appropriate cells in the unfolding.

\end{problem}
