\begin{problem}{Кубик}
{\textsl{standard input}}{\textsl{standard output}}
{1 second}{512 mebibytes}{}

Основное, что надо заметить в этой задаче --- что гексамино можно сгибать в разные стороны. При этом получаются два кубика, отличающихся друг от друга только взаимной перестановкой одной пары противоположных цифр (так как поворот на 180 градусов переставляет две пары противоположных цифр, то перестановка всех трёх пар эквивалентна перестановке одной пары). 

Далее существует несколько методов решения. 

Один из них состоит из следующих шагов:

\begin{enumerate}
\item Устанавливаем кубик нижней стороной на соответствующую клетку гексамино

\item Для каждого из четырёх поворотов кубика вокруг своей оси начинаем перекатывать его по гексамино до тех пор, пока возможно перекатить кубик так, чтобы нижняя грань попала на клетку с соответствующим номером. 

\item Если для какого-то из поворотов все 6 клеток посещены, то ответ ``\texttt{Yes}''. 

\item Если нет, то меняем местами цифры на правой и левой гранях и повторяем процедуру с шага 2. Если и в этом случае ни при одном из поворотов не посещены все 6 клеток, то ответ --- ``\texttt{No}''.
\end{enumerate}

Также существует возможность предпросчитать все 11 развёрток и затем проверять совпадение гексамино (поворачивая его в стандартное положение),
а в случае совпадения --- соответствие пар чисел, использованных для нумерации противоположных сторон кубика, с теми же парами в развёртке.

\end{problem}

