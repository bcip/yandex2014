\begin{problem}{НЛО}
{\textsl{standard input}}{\textsl{standard output}}
{1 second}{512 mebibytes}{}

Это самая простая задача набора.
Очевидно, что размерность параллелепипеда не меньше количества
различных длин рёбер.
Также заметим, что в $K$-мерном параллелепипеде $2^{K-1}$ рёбер одного типа,
то есть в случае наличия $X$ одинаковых рёбер для них будет использовано
$[(X-1)/2^{K-1}]+1$ измерение.

Поэтому в порядке возрастания перебираем все размерности,
пока вычисленное по указанной выше формуле суммарное количество
использованных измерений не будет меньше или равно текущей размерности.
Так как при $K=21$ рёбер одного типа будет $2^{20}>10^6$,
то перебор будет небольшим.

\end{problem}
