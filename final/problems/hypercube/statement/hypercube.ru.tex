\begin{problem}{НЛО}{stdin}{stdout}{1 секунда}{512 мегабайт}

Ночью в озеро, расположенное неподалёку от Васиного дома, упал НЛО.

НЛО представлял собой каркас $N$-мерного прямоугольного параллелепипеда, собранный из титановых рёбер, причём все рёбра имели целые длины. В момент крушения
соединения между рёбрами разрушились, в отличие от самих рёбер, которые оказались на дне и на берегу озера.

С утра Вася пришёл на берег озера и нашёл $K$ титановых стержней. Предполагая, что эти стержни являются рёбрами каркаса НЛО и что, возможно,
часть рёбер ещё находится под водой, определите наименьшую возможную размерность пространства $N$, из которого прибыл НЛО.


\InputFile
Первая строка ввода содержит целое число  $K$ ($1 \le K \le 10^6$) --- количество найденных Васей стержней. 
Вторая строка содержит $K$ целых чисел $a_i$ ($1 \le a_i \le 10^6$) --- длины стержней, разделённые пробелами.


\OutputFile
Выведите одно целое число --- наименьшую возможную размерность пространства $N$, из которого прибыл НЛО.

\Examples

\begin{example}
\exmp{3
1 2 3
}{3
}%
\exmp{3
1 2 2
}{2
}%
\end{example}

\end{problem}
